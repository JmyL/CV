%%%%%%%%%%%%%%%%%
% This is an sample CV template created using altacv.cls
% (v1.7, 9 August 2023) written by LianTze Lim (liantze@gmail.com). Compiles with pdfLaTeX, XeLaTeX and LuaLaTeX.
%
%% It may be distributed and/or modified under the
%% conditions of the LaTeX Project Public License, either version 1.3
%% of this license or (at your option) any later version.
%% The latest version of this license is in
%%    http://www.latex-project.org/lppl.txt
%% and version 1.3 or later is part of all distributions of LaTeX
%% version 2003/12/01 or later.
%%%%%%%%%%%%%%%%

%% Use the "normalphoto" option if you want a normal photo instead of cropped to a circle
% \documentclass[10pt,a4paper,normalphoto]{altacv}

\documentclass[10pt,a4paper,ragged2e,withhyper]{altacv}
%% AltaCV uses the fontawesome5 and packages.
%% See http://texdoc.net/pkg/fontawesome5 for full list of symbols.

% Change the page layout if you need to
\geometry{left=1.25cm,right=1.25cm,top=1.5cm,bottom=1.5cm,columnsep=1.2cm}

% The paracol package lets you typeset columns of text in parallel
\usepackage{paracol}

% Change the font if you want to, depending on whether
% you're using pdflatex or xelatex/lualatex
% WHEN COMPILING WITH XELATEX PLEASE USE
% xelatex -shell-escape -output-driver="xdvipdfmx -z 0" sample.tex
\ifxetexorluatex
  % If using xelatex or lualatex:
  \setmainfont{Roboto Slab}
  \setsansfont{Lato}
  \renewcommand{\familydefault}{\sfdefault}
\else
  % If using pdflatex:
  \usepackage[rm]{roboto}
  \usepackage[defaultsans]{lato}
  % \usepackage{sourcesanspro}
  \renewcommand{\familydefault}{\sfdefault}
\fi

% Change the colours if you want to
\definecolor{SlateGrey}{HTML}{2E2E2E}
\definecolor{LightGrey}{HTML}{666666}
\definecolor{DarkPastelRed}{HTML}{450808}
\definecolor{PastelRed}{HTML}{8F0D0D}
\definecolor{GoldenEarth}{HTML}{E7D192}
% \colorlet{name}{black}
% \colorlet{tagline}{PastelRed}
% \colorlet{heading}{DarkPastelRed}
% \colorlet{headingrule}{GoldenEarth}
% \colorlet{subheading}{PastelRed}
% \colorlet{accent}{PastelRed}
% \colorlet{emphasis}{SlateGrey}
% \colorlet{body}{LightGrey}

\definecolor{SlateGrey}{HTML}{2E2E2E}
\definecolor{LightGrey}{HTML}{666666}
\definecolor{DarkPastelBlue}{HTML}{0A4D8E}
\definecolor{PastelBlue}{HTML}{1C6BA0}
\definecolor{GoldenEarth}{HTML}{E7D192}
\colorlet{name}{black}
\colorlet{tagline}{PastelBlue}
\colorlet{heading}{DarkPastelBlue}
\colorlet{headingrule}{GoldenEarth}
\colorlet{subheading}{PastelBlue}
\colorlet{accent}{PastelBlue}
\colorlet{emphasis}{SlateGrey}
\colorlet{body}{LightGrey}

% Change some fonts, if necessary
\renewcommand{\namefont}{\Huge\rmfamily\bfseries}
\renewcommand{\personalinfofont}{\footnotesize}
\renewcommand{\cvsectionfont}{\LARGE\rmfamily\bfseries}
\renewcommand{\cvsubsectionfont}{\large\bfseries}


% Change the bullets for itemize and rating marker
% for \cvskill if you want to
\renewcommand{\cvItemMarker}{{\small\textbullet}}
\renewcommand{\cvRatingMarker}{\faCircle}
% ...and the markers for the date/location for \cvevent
% \renewcommand{\cvDateMarker}{\faCalendar*[regular]}
% \renewcommand{\cvLocationMarker}{\faMapMarker*}


% If your CV/résumé is in a language other than English,
% then you probably want to change these so that when you
% copy-paste from the PDF or run pdftotext, the location
% and date marker icons for \cvevent will paste as correct
% translations. For example Spanish:
% \renewcommand{\locationname}{Ubicación}
% \renewcommand{\datename}{Fecha}


%% Use (and optionally edit if necessary) this .tex if you
%% want to use an author-year reference style like APA(6)
%% for your publication list
% \input{pubs-authoryear.tex}

%% Use (and optionally edit if necessary) this .tex if you
%% want an originally numerical reference style like IEEE
%% for your publication list
\usepackage[backend=biber,style=ieee,sorting=ydnt,defernumbers=true]{biblatex}
%% For removing numbering entirely when using a numeric style
\setlength{\bibhang}{1.25em}
\DeclareFieldFormat{labelnumberwidth}{\makebox[\bibhang][l]{\itemmarker}}
\setlength{\biblabelsep}{0pt}
\defbibheading{pubtype}{\cvsubsection{#1}}
\renewcommand{\bibsetup}{\vspace*{-\baselineskip}}
\AtEveryBibitem{%
  \iffieldundef{doi}{}{\clearfield{url}}%
}



%% sample.bib contains your publications
% \addbibresource{sample.bib}

\begin{document}
\name{Sungsik Nam}
\tagline{Engineer specialized in Deep Learning and Embedded C/C++\\with Process/Team Management Experience in Mobile Industries}
%% You can add multiple photos on the left or right
\photoR{3.4cm}{profile}
% \photoL{2.5cm}{Yacht_High,Suitcase_High}

\personalinfo{%
  % Not all of these are required!
  \email{jmyl@me.com}
  \phone{+49 151 4139 7269}
  \mailaddress{Driesener Str. 8A, 10439 Berlin}
  \location{Berlin, Germany}
  \homepage{jmyl.github.io}
  \github{jmyl}
  \linkedin{sungsik-nam}
  % \twitter{@twitterhandle}
  % \orcid{0000-0000-0000-0000}
  %% You can add your own arbitrary detail with
  %% \printinfo{symbol}{detail}[optional hyperlink prefix]
  % \printinfo{\faPaw}{Hey ho!}[https://example.com/]

  %% Or you can declare your own field with
  %% \NewInfoFiled{fieldname}{symbol}[optional hyperlink prefix] and use it:
  % \NewInfoField{gitlab}{\faGitlab}[https://gitlab.com/]
  % \gitlab{your_id}
  %%
  %% For services and platforms like Mastodon where there isn't a
  %% straightforward relation between the user ID/nickname and the hyperlink,
  %% you can use \printinfo directly e.g.
  % \printinfo{\faMastodon}{@username@instace}[https://instance.url/@username]
  %% But if you absolutely want to create new dedicated info fields for
  %% such platforms, then use \NewInfoField* with a star:
  % \NewInfoField*{mastodon}{\faMastodon}
  %% then you can use \mastodon, with TWO arguments where the 2nd argument is
  %% the full hyperlink.
  % \mastodon{@username@instance}{https://instance.url/@username}
}

\makecvheader
%% Depending on your tastes, you may want to make fonts of itemize environments slightly smaller
% \AtBeginEnvironment{itemize}{\small}

%% Set the left/right column width ratio to 6:4.
\columnratio{0.57}

% Start a 2-column paracol. Both the left and right columns will automatically
% break across pages if things get too long.
\begin{paracol}{2}
\cvsection{Experience}

  \cvevent{Deep Learning Engineer}{VUERON}{Feb 2023 -- Present}{Munich, Germany (Home Office)}
  \begin{itemize}
    \item Chose, trained LiDAR object detection network
    \item Implemented and evaluated knowledge distillation
    \item Implemented real-time end-to-end object detection
  \end{itemize}

  \divider

  \cvevent{Principal Software Engineer}{CREATZ}{Jun 2020 -- Jan 2023}{Seoul, South Korea}
  \begin{itemize}
    \item Managed a research for detecting the rotation of golf ball
    \item Designed image processing sequences and kernels
    % \item Developed IP-cam streaming \& recording library
  \end{itemize}

  \divider

  \cvevent{Principal Engineer}{MELFAS}{Jul 2005 -- Aug 2019}{Pangyo, South Korea}
  \begin{itemize}
    \item Managed projects, providing touchscreen IC to Samsung
    \item Led a team of 20 engineers
    \item Invented key algorithms and panel structures,\\
      \smallskip
      \textbf{\color{emphasis}filed 7 patents with 6 as the first inventor}
  \end{itemize}

\medskip

\cvsection{Projects}

% \item 3D OD(VoxelNeXt) to 2D(CenterPoint) knowledge distillation 
  \cvproject{Research: VoxelNeXt to CenterPoint KD in Python}{}{}{}
  \begin{itemize}
    \item mAP increased by 0.5
    \item Wrote knowledge distillation code with OpenPCDet
    \item Wrote network-comparison platform with Pandas \& Quarto
  \end{itemize}

  \smallskip
 
  % \cvproject{Research: OD with Conventional Methods in C}{}{}{}
  % \begin{itemize}
  %   \item Wrote ground-plane detection and object clustering algorithm
  %   \item Wrote and enhanced bounding-box determination algorithm
  % \end{itemize}
  %
  % \smallskip
  %
  \cvproject{CenterPoint on TI board with C++}{}{}{}
  \begin{itemize}
    \item Evaluated 8-bit quantized model (on going)
    \item Made network compilable by:
    \smallskip
    \begin{itemize}
      \item fusing MLP+BatchNorm into 1x1 Conv
      \item replace unsupported layer with supported one% like 1x1 ConvTranspose \& Tile
  \end{itemize}
    \item Built library for TDA4VM with ONNXRuntime in C++
  \end{itemize}

  \smallskip

  \cvproject{CenterPoint on NVIDIA/Intel Platform with C++}{}{}{}
  \begin{itemize}{}{}{}
    \item Developed DL inference code with TensorRT/OpenVINO
    \item Wrote voxelize, encode, scatter, NMS with CUDA/SYCL
  \end{itemize}

  \smallskip

  \cvproject{Research: Ball Spin Calculation Using Dimples in Python}{}{}{}
  \begin{itemize}
    \item Designed and implemented test loop with 3D ball model
    \item Designed 2D convolution kernels and wrote image processing code to determine center position of dimples
  \end{itemize}

  \smallskip
 
  % \cvproject{Protocol Converter: USB3 to GigE-V in C}{}{}{}
  % \begin{itemize}
  %   \item On Raspberry Pi
  %   \item Use opensource code to emulate GigE-V host
  % \end{itemize}
  %
  % \smallskip
 
  % \cvproject{IP Camera Control/Streaming in C++}{}{}{}
  % \begin{itemize}
  %   \item With LIVE555 library
  %   \item Controled an IP camera via CGI API
  %   \item Received RTSP streams, stored the streams
  % \end{itemize}
  %
  % \smallskip
  \medskip
 
  \cvproject{Touchscreen Research with Clients}{}{}{}
  \begin{itemize}
    \item Led on-cell touchscreen research With Samsung Display
    \item Led active-type digitizer research With LG Display
    \item Led single-layer touchscreen proejct with SHARP
  \end{itemize}

  \smallskip

  \cvproject{Touch IC firmware development}{}{}{}
  \begin{itemize}
    \item Brought up 15 ICs in MELFAS
    \item Wrote bootloader and real-time (120Hz) bare-metal firmware for Cortex-M series IC
    \item Designed tuning method of analog peripherals
  \end{itemize}


\cvsection{Patents}

  \cvevent{U.S. Patent US9921693B2}{Method for Processing Multi-Touch}{Mar 20, 2018}{also in China \& Korea}

  \begin{itemize}
  \item Distinguishing between single finger with wide area and double fingers with narrow spacing
  \item Satisfying Microsoft Windows 10 HLK and Samsung's high-level criteria
  \item Adapted as a touch solution for Samsung's high-end models and LG Display's Windows laptops 
  \end{itemize}
  \medskip

  \cvevent{U.S. Patent US9164635B2}{Method of Controlling Noise of Touch}{Oct 20, 2015}{also in China}
  \begin{itemize}
  \item Invented sensing frequency drifting method, using anti-aliasing positive way
  \item Adapted as a touch solution for Samsung Display's on-cell touchscreen buisiness
  \end{itemize}
  \medskip

  \cvevent{China Patent CN101925872B}{Touch Panel having a Split-Electrode}{Jun 25, 2014}{also in Japan \& Korea}

  \begin{itemize}
  \item Designed one-layer panel pattern for single touch
  \item Became the standard form for affordable single-touch support touchscreens
  \end{itemize}
  \medskip

  \cvevent{U.S. Patent US9690436B2}{Touch Panel using Single-Layer Pattern}{Jun 27, 2017}{also in China}

  \begin{itemize}
  \item Designed one-layer panel pattern for multi touch
  \item Invented a sensing method to detect pad with big resistance
  \item Adapted as a standard single-layer pattern for MELFAS's ITO touchscreen panels
  \end{itemize}
  \medskip

  \cvevent{U.S. Patent US11221715B2}{Method for Measuring Touch Pressure With Touch Pannel}{Jan 11, 2022}{also in China \& Korea}
  \medskip

  \cvevent{U.S. Patent US10067611B2}{Apparatus and Method for Detecting a Touch with Compensating Position Distortion due to the Panel Bending}{Sep 4, 2018}{also in Korea}
  \medskip

  \cvevent{China Patent CN101983371B}{Touch Panel with Improved Edge Position Recognition Characteristics, by Shiwon Ryu and I}{Apr 9, 2014}{also in Korea}
  \medskip

% \cvsection{A Day of My Life}
%
% % Adapted from @Jake's answer from http://tex.stackexchange.com/a/82729/226
% % \wheelchart{outer radius}{inner radius}{
% % comma-separated list of value/text width/color/detail}
% \wheelchart{1.5cm}{0.5cm}{%
%   6/8em/accent!30/{Sleep,\\beautiful sleep},
%   3/8em/accent!40/Hopeful novelist by night,
%   8/8em/accent!60/Daytime job,
%   2/10em/accent/Sports and relaxation,
%   5/6em/accent!20/Spending time with family
% }

% use ONLY \newpage if you want to force a page break for
% ONLY the current column
\newpage

% \cvsection{Publications}
%
% %% Specify your last name(s) and first name(s) as given in the .bib to automatically bold your own name in the publications list.
% %% One caveat: You need to write \bibnamedelima where there's a space in your name for this to work properly; or write \bibnamedelimi if you use initials in the .bib
% %% You can specify multiple names, especially if you have changed your name or if you need to highlight multiple authors.
% \mynames{Lim/Lian\bibnamedelima Tze,
%   Wong/Lian\bibnamedelima Tze,
%   Lim/Tracy,
%   Lim/L.\bibnamedelimi T.}
% %% MAKE SURE THERE IS NO SPACE AFTER THE FINAL NAME IN YOUR \mynames LIST
%
% \nocite{*}
%
% \printbibliography[heading=pubtype,title={\printinfo{\faBook}{Books}},type=book]
%
% \divider
%
% \printbibliography[heading=pubtype,title={\printinfo{\faFile*[regular]}{Journal Articles}},type=article]
%
% \divider
%
% \printbibliography[heading=pubtype,title={\printinfo{\faUsers}{Conference Proceedings}},type=inproceedings]

%% Switch to the right column. This will now automatically move to the second
%% page if the content is too long.
\switchcolumn

\cvsection{My Dev Philosophy}

\begin{quote}
\normalsize
``You can’t manage what you can’t measure.''\\
``People over assets.''
\end{quote}

\cvsection{Most Proud of}

\cvachievement{\faTrophy}{Deep Understanding of DL Networks}{Reviewed many papers on 2D \& 3D OD domain, contributed choosing DL networkfor 3D object detection}

\cvachievement{\faTrophy}{Open to New Technology}{Implemented End-to-end OD with SYCL \& OpenVINO on Intel Platform, reducing latency to 24\% of its original level}

\cvachievement{\faTrophy}{Creative Problem Solving}{As a pioneer in the touch screen industry, invented key algorithms, including Patent US9921693B2}

\cvachievement{\faTrophy}{Management Experience}{Managed 20 engineers as a tech lead, managed S/W development process, with satisfying Samsung for 14 years}

\cvachievement{\faTrophy}{Proficiency in Handling Dev Environment}{Set up dev-containers, CI/CD, c++ unit test with googletest and CTest, and CMake build system in a modern and efficient way}

\cvsection{Strengths}

Languages
\smallskip

\cvtag{C/C++}
\cvtag{Python}
\cvtag{SYCL}
\cvtag{CUDA}\\
\cvtag{R}
\cvtag{Matlab}

\smallskip
Python Libraries
\smallskip

\cvtag{PyTorch}
\cvtag{Numpy}
\cvtag{Pandas}
\cvtag{Seaborn}
\cvtag{ONNX-GraphSurgeon}

\smallskip
C++ Libraries
\smallskip

\cvtag{C++ stdlib}
\cvtag{containers}
\cvtag{ONNXRuntime}
\cvtag{OpenVINO}
\cvtag{OpenCV}
\cvtag{OpenMP}

\smallskip
CI/CD
\smallskip

\cvtag{GoogleTest}
\cvtag{CTest}
\cvtag{CDash}
\cvtag{Gitlab CI}\\
\cvtag{Github Actions}
\cvtag{Bitbucket Pipeline}

\smallskip
Etc.
\smallskip

\cvtag{Docker}
\cvtag{Docker Compose}
\cvtag{Quarto}\\
\cvtag{Modern CMake}
\cvtag{Bash Script}
\cvtag{Neovim}

\cvsection{Favorite Books}

\begin{itemize}
\item Pragmatic Programmer - Thomas \& Hunt
\item How to Know a Person - Brooks
\end{itemize}

\cvsection{Languages}

\cvskill{Korean - native}{5}
\medskip
\cvskill{English - C1}{4}
\medskip
\cvskill{German - C1}{4} %% Supports X.5 values.

%% Yeah I didn't spend too much time making all the
%% spacing consistent... sorry. Use \smallskip, \medskip,
%% \bigskip, \vspace etc to make adjustments.
\medskip

\cvsection{Education}

\cvevent{B.Sc.\ in Statistics (3 Semesters)}{TU Dortmund}{Oct 2021 -- Feb 2023}{}

Learned mathematics and statistics to fully understand DL networks

\medskip

\cvevent{B.Eng.\ in Electrical and Electronics Engineering}{Seoul National University}{Mar 2002 -- Feb 2006}{}

\medskip


% \divider

% \cvsection{Referees}
%
% % \cvref{name}{email}{mailing address}
% \cvref{Prof.\ Alpha Beta}{Institute}{a.beta@university.edu}
% {Address Line 1\\Address line 2}
%
% \divider
%
% \cvref{Prof.\ Gamma Delta}{Institute}{g.delta@university.edu}
% {Address Line 1\\Address line 2}


\end{paracol}


\end{document}
